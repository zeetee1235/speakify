\documentclass[aspectratio=169,11pt]{beamer}

% ---- Theme ----
\usetheme{Madrid}
\usecolortheme{dove}
\setbeamertemplate{navigation symbols}{}

% Custom colors for a softer gray theme - override all blues
\setbeamercolor{structure}{fg=black!70}
\setbeamercolor{frametitle}{bg=black!10,fg=black!80}
\setbeamercolor{title}{fg=black!80}
\setbeamercolor{item}{fg=black!60}
\setbeamercolor{title in head/foot}{fg=black}
\setbeamercolor{author in head/foot}{fg=black!70}
\setbeamercolor{date in head/foot}{fg=black!70}
\setbeamercolor{section in head/foot}{bg=black!15,fg=black}
\setbeamercolor{subsection in head/foot}{bg=black!10,fg=black!70}

% ---- Packages ----
\usepackage[T1]{fontenc}
\usepackage[utf8]{inputenc}
\usepackage{kotex}
\usepackage{amsmath,amssymb,amsthm}
\usepackage{mathtools}
\usepackage{booktabs}
\usepackage{graphicx}
\usepackage{xcolor}
\usepackage{hyperref}

% algorithm
\usepackage{algorithm}
\usepackage{algpseudocode}

% ---- Hyperref ----
\hypersetup{colorlinks=true,linkcolor=blue,citecolor=blue,urlcolor=blue}

% ---- Title ----
\title[Speakify]{\textbf{Speakify 픽셀 재배치 기반 모핑}\\[0.3em]
\large 색채-공간 결합 비용함수와\\
스플랫 정규화 렌더링의 통합적 고찰}
\author{zeetee}
\date{\today}

\begin{document}

% -------------------- Title --------------------
\begin{frame}
  \titlepage
\end{frame}

% -------------------- Abstract --------------------
\begin{frame}{Abstract}
본 문서는 Speakify의 핵심 기제를 이루는 두 축, 즉
\emph{(i) 픽셀 매칭 최적화}와 \emph{(ii) 프레임 생성 모핑 렌더링}을 수학적 관점에서 정리한다.

\medskip
\begin{itemize}
  \item 색상 유사성과 공간 근접성을 결합한 휴리스틱 비용함수 정의
  \item 랜덤 스왑 기반의 탐욕적 국소탐색으로 대입(assignment) 상태를 점진 개선
  \item cubic ease-in-out 시간 보간, $3\times3$ 스플랫 분배, 누적 가중치 정규화,
        BFS 기반 홀 보정으로 이어지는 렌더링 체계 분석
\end{itemize}

\medskip
해당 설계는 구현 단순성과 시각적 안정성 사이의 따잇을 지향하며,
특히 ``슝좍'' 및 순수 따잇 억제에 실용적 성능을 보인다.
\end{frame}

% -------------------- Intro --------------------
\section{서론}

\begin{frame}{서론}
이미지 모핑의 실용적 성능은
\textbf{``어떤 픽셀이 어디로 이동하는가''}라는 대응 문제와
\textbf{``이동 과정을 어떻게 시각화하는가''}라는 렌더링 문제의 결합에서 결정된다.

\medskip
Speakify는 전역 최적해를 직접 구하는 고비용 접근 대신,
확률적 제안과 탐욕적 수용 규칙을 통해 저비용의 고품질 근사해를 구축한다.

\medskip
본 문서의 목적은 이 파이프라인의 구조를 학술적 형식으로 명료화하고,
각 설계 선택이 결과 영상의 질감에 미치는 영향을 정리하는 데 있다.
\end{frame}

% -------------------- Algorithm A --------------------
\section{알고리즘 A: 픽셀 매칭 최적화}

\subsection{비용함수 정의}

\begin{frame}{알고리즘 A: 비용함수 정의}
소스 픽셀 $p_i$와 타깃 위치 $q_{\pi(i)}$ 사이의 비용을 아래와 같이 정의한다.
\begin{align}
D_{\text{color}}(i) &= (\Delta R_i)^2 + (\Delta G_i)^2 + (\Delta B_i)^2, \\
D_{\text{spatial}}(i) &= (\Delta x_i)^2 + (\Delta y_i)^2, \\
C(i) &= w_c\,D_{\text{color}}(i) + w_s\,D_{\text{spatial}}(i).
\end{align}

총 목적함수는
\begin{equation}
\mathcal{L}(\pi) = \sum_i C(i)
\end{equation}
로 기술된다.

\medskip
이 구성은 색상 일치도만을 추구하는 단순 대입과 달리
장거리 이동에 패널티를 부여함으로써, 모핑 과정의 구조적 연속성을 보존한다.
\end{frame}

\subsection{최적화 전략: 랜덤 스왑 기반 탐욕적 국소탐색}

\begin{frame}{알고리즘 A: 최적화 전략}
구현은 ``genetic algorithm''이라는 표기와 달리,
실질적으로는 \emph{확률적 제안 + 탐욕적 수용} 형태의 hill-climbing에 가깝다.

\medskip
초기 상태는 항등 대입(identity assignment)이며,
반복적으로 두 위치 $(a,b)$를 선택해 swap 제안을 평가한다.
\begin{itemize}
  \item $a$는 전역에서 무작위 선택
  \item $b$는 반경 \texttt{max\_dist} 내에서 무작위 선택
  \item swap 후 $\Delta\mathcal{L}<0$이면 수용, 아니면 기각
\end{itemize}

\medskip
탐색 반경은 반복마다 $0.99$배로 축소되어
초기 전역 탐색에서 후기 미세 조정으로 자연스럽게 이행한다.
또한 \texttt{max\_dist}<4 이고 실질 개선이 희박하면 종료한다.
\end{frame}

\begin{frame}{알고리즘 A: 절차 요약 (의사코드)}
\begin{algorithm}[H]
\caption{Random Swap Greedy Local Search}
\begin{algorithmic}[1]
\State Initialize assignment $\pi \leftarrow$ identity
\State Set \texttt{max\_dist} to initial radius
\While{\texttt{max\_dist} $\ge 4$ and improvements exist}
  \State pick $a$ uniformly at random
  \State pick $b$ uniformly at random within radius \texttt{max\_dist} around $a$
  \State compute $\Delta \mathcal{L}$ for swapping $\pi(a)$ and $\pi(b)$
  \If{$\Delta \mathcal{L}<0$}
    \State accept swap
  \Else
    \State reject swap
  \EndIf
  \State \texttt{max\_dist} $\leftarrow 0.99 \cdot$ \texttt{max\_dist}
\EndWhile
\end{algorithmic}
\end{algorithm}
\end{frame}

\subsection{재현성과 제어 변수}

\begin{frame}{알고리즘 A: 재현성과 제어 변수}
\begin{itemize}
  \item 난수 시드: \texttt{12345} (결과 재현성 확보)
  \item CLI 매개변수: \texttt{--proximity}로 $w_s$ 조절 (기본값 13)
  \item WASM 경로: proximity 값 13으로 고정
\end{itemize}

\medskip
본 구조는 계산 효율이 높고 구현이 단순하나,
나쁜 이동을 확률적으로 허용하지 않으므로 국소해 정체 가능성이 존재한다.
\end{frame}

% -------------------- Algorithm B --------------------
\section{알고리즘 B: 프레임 생성 및 모핑 렌더링}

\subsection{시간 보간: cubic ease-in-out}

\begin{frame}{알고리즘 B: 시간 보간 (cubic ease-in-out)}
프레임 진행률 $t\in[0,1]$를 직접 사용하지 않고
$\phi(t)=\text{ease\_in\_out\_cubic}(t)$로 변환한다.

\medskip
이에 따라
\begin{itemize}
  \item 초기/말기 구간은 완만해지고
  \item 중간 구간은 가속되어
\end{itemize}
운동의 지각적 이질감을 완화한다.
\end{frame}

\subsection{위치 보간과 스플랫 분배}

\begin{frame}{알고리즘 B: 위치 보간과 스플랫 분배}
각 픽셀의 중간 좌표 $(f_x,f_y)$를 선형 보간으로 계산한 후,
단일 격자점에 할당하지 않고 주변 $3\times3$ 이웃으로
가중 분배(splat)한다.

\medskip
이는 다음 효과를 제공한다.
\begin{itemize}
  \item 비어 보이는 홀(hole) 감소
  \item aliasing 및 프레임 간 깜빡임 완화
  \item 고속 이동 구간의 시각적 연속성 향상
\end{itemize}
\end{frame}

\subsection{정규화 및 홀 채움}

\begin{frame}{알고리즘 B: 정규화 및 홀 채움}
렌더 버퍼의 누적 가중치 $W$에 대해 색상 합을 정규화하여 최종 픽셀을 얻는다.

\medskip
아직 채워지지 않은 지점은 채워진 픽셀 집합으로부터
4-이웃 BFS 확산을 수행해 보정한다.

\medskip
이는 스플랫 이후 잔존 공백을 안정적으로 메우는
후처리 단계로 기능한다.
\end{frame}

% -------------------- Discussion --------------------
\section{논의}

\begin{frame}{논의}
Speakify의 핵심은
\textbf{``비용함수에서의 공간 제약''}과
\textbf{``렌더링에서의 분산 누적''}의 결합이다.

\medskip
\begin{itemize}
  \item 전자는 기하학적 파열을 억제하고,
  \item 후자는 중간 프레임의 밀도 결손을 줄인다.
\end{itemize}

\medskip
즉, 최적화 단계와 렌더링 단계가 상보적으로 작동하여,
제한된 계산 자원 하에서도 높은 지각 품질의 모핑 결과를 생성한다.
\end{frame}

% -------------------- Conclusion --------------------
\section{결론}

\begin{frame}{결론}
본 파이프라인은 복잡한 전역 최적화 기법 없이도
색상 정합성과 공간 일관성을 동시에 추구하는 실용적 설계를 제시한다.

\medskip
특히 아래 요소는 결과 품질을 좌우하는 핵심 제어점이다.
\begin{itemize}
  \item \texttt{proximity\_importance} (즉 $w_s$)
  \item 반경 감쇠율
  \item 스플랫/정규화/홀보정 조합
\end{itemize}

\medskip
향후 확장으로는 확률적 황크 수용(예: get you crayon 계열) 도입,
PF(pumpkin friends) 초기화, 그리고 trick kernel 설계를 통해
mayo's PFNTR 민감도와 Pure tait 비율을 동시에 개선할 수 있다.
\end{frame}


\end{document}
